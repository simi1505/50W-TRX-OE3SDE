In this chapter, the MCU Code written with Arduino IDE is implemented. 
The internal hardware of the used MCU (ATmega328) is programmed on register level.
To control the OLED display, the external DAC and some sensors, external libraries are used.

\begin{lstlisting}[language=Arduino]
//Arduino Code for DRA818V FM TRX + 60W RA60H1317M1A
//This code is written by OE3SDE, Simon Dorrer!

//Define CPU-Clock-Speed (16MHz internal clock)
//#define F_CPU 16000000UL

//Define Baudrate for UART
//#define BAUD 9600UL

//Include Libraries
#include <avr/io.h>
#include <stdio.h>
#include <avr/interrupt.h>

//I2C
#include <Wire.h>

//MCP4728
#include <Adafruit_MCP4728.h>

//DS18B20
#include <OneWire.h> 
#include <DallasTemperature.h>

//Libs. for SD1306 OLED-Display
#include <Adafruit_GFX.h>               // Include core graphics library for the display
#include <Adafruit_SSD1306.h>           // Include Adafruit_SSD1306 library to drive the display
#include <Fonts/FreeMono9pt7b.h>        // Add a custom font
/*List of different fonts :
FreeMono9pt7b.h
FreeMonoBold9pt7b.h
FreeMonoBoldOblique9pt7b.h
FreeMonoOblique9pt7b.h  
*/
//----------------------------------------------------------------------------

//Definitions
//I/O Declaration
//1W FM TRX Board (DRA818V)
//#define TX        0     //PD0 / D0
//#define RX        1     //PD1 / D1
#define SW1         2     //PD2 / D2 - INT0
#define PTT_IN      3     //PD3 / D3 - INT1
#define PTT_OUT     7     //PD7 / D7
#define PD          0     //PB0 / D8
#define H_L         1     //PB1 / D9
#define HC05_TX     2     //PB2 / D10
#define HC05_RX     3     //PB3 / D11
#define E1          0     //PC0 / A0
#define E2          1     //PC1 / A1

//60W FM TRX Board (RA60H1317M1A)
#define PGA_SHDN    5     //PB5 / D13 (LOW --> PGA OFF, HIGH --> PGA ON) + RX Switch
#define TX_SW       4     //PB4 / D12 (HIGH TX Switch)
//#define T1_SW     7     //PD7 / D7  (switches T1 for TRX Switch - LOW for TX, HIGH for RX) --> see PTT_OUT D7
#define DS18B20_OW  6     //PD6 / D6  (One wire input of DS18B20 Temp. Sensor)
#define SWR_REV     7     //PC7 / A7  (ADC samples reverse voltage)
#define SWR_FWD     6     //PC6 / A6  (ADC samples forward voltage)
#define ADC_VDD     3     //PC3 / A3  (ADC samples VDD voltage) 

//I2C: OLED Display, MCP4728

//Makros
//RX or TX ATTENUATOR
#define RX_ATTENUATOR_CHANNEL  MCP4728_CHANNEL_A
#define TX_ATTENUATOR_CHANNEL  MCP4728_CHANNEL_B

//PA max. ratings
#define VDD_LOWER_LIMIT 10.7  //in V
#define VDD_UPPER_LIMIT 14.2  //in V
#define SWR_LOWER_LIMIT 0.5   
#define SWR_UPPER_LIMIT 2
#define HEAT_UPPER_LIMIT 45   //in degrees
//----------------------------------------------------------------------------

//Global variables / Objects
uint8_t TX = 0;  // 0... RX; 1... TX
uint8_t switched = 0;

//OLED
Adafruit_SSD1306 display(128, 64);    // Create display
volatile uint16_t timer1Count = 0;

//DRA818 Variables
typedef struct{
	float txFreq;         //TX-Frequency in MHz (134.0000 - 174.0000)
	float rxFreq;         //RX-Frequency in MHz (134.0000 - 174.0000)
	String txCTCSS;       //CTCSS frequency (0000 - 0038); 0000 = "no CTCSS" 
	String rxCTCSS;       //CTCSS frequency (0000 - 0038); 0000 = "no CTCSS" 
	uint8_t bw;           //Bandwith in KHz (0= 12.5KHz or 1= 25KHz)
	uint8_t squ;          //Squelch level  (0 - 8); 0 = "open" 
	uint8_t vol;
	uint8_t prf;
	uint8_t hpf;
	uint8_t lpf;
}DRA818;

DRA818 dra818 = {145.5000, 145.5000, "0000", "0000", 1, 4, 8, 0, 0, 0};

//DS18B20
OneWire oneWire(DS18B20_OW); 
DallasTemperature ds18b20(&oneWire);

//MCP4728
Adafruit_MCP4728 mcp4728;

//Rotary Encoder Variables
int counter = 0; 
int aState;
int aLastState;  

//RA60H1317M1A Variables
float txGain = 10;
float txAtt = 0;
float rxAtt = 0;
uint8_t monitorPA = 0;  //monitorPA = 1... PA is OK!
//----------------------------------------------------------------------------

//Subprograms
//Hardware Init Methods
void IO_Init()                  //initialize IO
{
	//Define INPUTs
	DDRD &= ~(1 << SW1);          //set PD2 (SW1) as Input
	PORTD |= (1 << SW1);          //activate Pull-Up-R at PD2 (SW1)    
	
	DDRD &= ~(1 << PTT_IN);       //set PD3 (PTT-IN) as Input
	PORTD |= (1 << PTT_IN);       //activate Pull-Up-R at PD3 (PTT_IN)     
	
	DDRC &= ~(1 << E1);           //set PC0 (E1) as Input    
	DDRC &= ~(1 << E2);           //set PC1 (E2) as Input
	
	DDRC &= ~(1 << ADC_VDD);      //set PC3 (ADC_VDD) as Input  
	//PORTC |= (1 << ADC_VDD);      //activate Pull-Up-R at PC3 (ADC_VDD)
	
	DDRC &= ~(1 << SWR_FWD);      //set PC6 (SWR_FWD) as Input  
	PORTC |= (1 << SWR_FWD);      //activate Pull-Up-R at PC6 (SWR_FWD)
	
	DDRC &= ~(1 << SWR_REV);      //set PC7 (SWR_REV) as Input  
	PORTC |= (1 << SWR_REV);      //activate Pull-Up-R at PC7 (SWR_REV)
	
	//Define OUTPUTs
	DDRD |= (1 << PTT_OUT);       //set PD7 (PTT_OUT) as Output
	DDRB |= (1 << PD);            //set PB0 (PD) as Output
	DDRB |= (1 << H_L);           //set PB1 (H_L) as Output
	
	DDRB |= (1 << TX_SW);         //set PB4 (TX_Switch) as Output
	DDRB |= (1 << PGA_SHDN);      //set PB5 (PGA_SHDN) as Output
}

void Timer1_COMPA_Init()
{
	//Control Registers
	TCCR1A = 0; TCCR1B = 0; TCCR1C = 0; //Reset
	
	// set CTC mode
	//TCCR1A |= (1 << WGM10);
	//TCCR1A |= (1 << WGM11);
	TCCR1B |= (1 << WGM12);
	
	// Set Prescaler 1024 (TCCR1B = (1 << WGM12) | (0x5 << CS10);)
	TCCR1B |= (1 << CS12);
	TCCR1B &= ~(1 << CS11);
	TCCR1B |= (1 << CS10);
	
	// initialize compare value
	OCR1A = 16;
	
	// enable timer compare interrupt
	TIMSK1 |= (1 << OCIE1A);
}

void Timer2_COMPA_Init()
{
	//Timer2 settings
	TCCR2A = 0x00; TCCR2B = 0x00; //Reset
	
	//Enable Timer2 CTC Mode
	//TCCR2A |= (1 << WGM20);
	TCCR2A |= (1 << WGM21);
	//TCCR2B |= (1 << WGM22);
	
	//Prescaler
	// 1024 prescaling for Timer2 (TCCR2B = (0x7 << CS20);)
	TCCR2B |= (1 << CS20);
	TCCR2B |= (1 << CS21);
	TCCR2B |= (1 << CS22);
	
	//Initialize compare value
	OCR2A = 0;
	
	//initialize TIMER0-Counter
	//TCNT2 = 0; // set counter value FORMEL: x = maximaler Zaehlwert - ((CPUtakt/PRESCALER)/ gesuchte Frequenz)
	
	//disable Timer compare interrupt
	TIMSK2 &= ~(1 << OCIE2A);
}

void INT0_Init()
{
	//enable Interrupt
	EIMSK |= (1 << INT0); //Ext. Int0 ein
	
	//Set falling Edge Interrupt (EICRA |= (0x2 << ISC00);)
	EICRA &= ~(1 << ISC00);
	EICRA |= (1 << ISC01);
}
void INT1_Init()
{
	//enable Interrupt
	EIMSK |= (1 << INT1); //Ext. Int1 ein
	
	//Set rising & falling Edge Interrupt (EICRA |= (0x2 << ISC10);)
	EICRA |= (1 << ISC10);
	EICRA &= ~(1 << ISC11);
}

void ADC_Init()
{
	//ADC Setup
	//Set Reference Voltage (VCC = VREF)
	ADMUX &= ~(1 << REFS1);
	ADMUX |= (1 << REFS0);
	
	//Prescaler --> 128 (50kHz - 200kHz)
	ADCSRA |= (1 << ADPS0);
	ADCSRA |= (1 << ADPS1);
	ADCSRA |= (1 << ADPS2);
	
	ADCSRA |= (1 << ADEN); //Enable ADC
	
	//Dummy Readout to warm up the ADC
	ADCSRA |= (1<<ADSC);          //Start ADC conclusion
	while (ADCSRA & (1<<ADSC)){}  //wait until ADC is ready
	(void) ADC;
}
uint16_t ADC_ReadValue(uint8_t channel)
{
	ADMUX = (ADMUX & ~(0x1F)) | (channel & 0x1F); //which ADCx? Bit Mask to secure ADMUX settings in Init() function
	
	ADCSRA |= (1<<ADSC);          //Start ADC conclusion
	while (ADCSRA & (1<<ADSC)){}  //wait until ADC is ready
	
	return ADC;                   //return the ADC value
}

//DRA818V Methods
void DRA818V_setGroup()
{
	Serial.print("AT+DMOSETGROUP=");         // begin message
	Serial.print(dra818.bw);
	Serial.print(",");
	Serial.print(dra818.txFreq, 4);
	Serial.print(",");
	Serial.print(dra818.rxFreq, 4);
	Serial.print(",");
	Serial.print(dra818.txCTCSS);
	Serial.print(",");
	Serial.print(dra818.squ);
	Serial.print(",");
	Serial.println(dra818.rxCTCSS);
}
void DRA818V_setVolume()
{
	Serial.print("AT+DMOSETVOLUME=");
	Serial.println(dra818.vol);
}
void DRA818V_setFilter()
{
	Serial.print("AT+SETFILTER=");
	Serial.print(dra818.prf);
	Serial.print(",");
	Serial.print(dra818.hpf);
	Serial.print(",");
	Serial.println(dra818.lpf);
}
void DRA818V_Init()             // initialize DRA818V
{
	PORTB &= ~(1 << PD);
	delay(2000);
	
	//I/O
	PORTD |= (1 << PTT_OUT);      // set PD7 (PTT_OUT) HIGH at the beginning (RX-Mode)
	PORTB |= (1 << PD);           // set PB0 (PD) HIGH at the beginning (Normal-Mode)
	PORTB &= ~(1 << H_L);         // set PB1 (H/L) LOW at the beginning (LOW-Power = 0.5W)
	
	//UART
	Serial.begin(9600);
	delay(10);
	DRA818V_setGroup();
	delay(500);
	DRA818V_setVolume();
	delay(500);
	DRA818V_setFilter();
	delay(500);
}

//RA60H1317M1A Methods
float getVDD()      //Get VDD Voltage (10.8VDC to 13.6VDC)
{
	float vdd = 0;
	// #PJN: You may spend a LOT of time in this loop => make ADC reading asynchronous
	
	vdd = ADC_ReadValue(ADC_VDD) * 5.0 / 1024.0;
	
	return((vdd * ((4700 + 18000) / 4700)) + 2.2);  //returns supply voltage without voltage division (magic numbers equals the voltage divider resistor values)
}
float getSWR()      //Get SWR of output load (should be between 0.5 and 2)
{
	float rev = 0;
	float fwd = 0;
	// #PJN: You may spend just more time in this loop => make ADC reading asynchronous
	for(int i = 0; i < 10; i++)   //10 iterations for a more precise result
	{
		rev = rev + ADC_ReadValue(SWR_REV);
		fwd = fwd + ADC_ReadValue(SWR_FWD);
	}
	rev = rev / 10 / 1024 * 5;
	fwd = fwd / 10 / 1024 * 5;
	
	//SWR Calculation
	return((fwd + rev) / (fwd - rev));
}
float getPWR()
{
	float fwd = 0;
	for(int i = 0; i < 10; i++)   //10 iterations for a more precise result
	{
		fwd = fwd + ADC_ReadValue(SWR_FWD);
	}
	fwd = fwd / 10 / 1024 * 5;
	
	//PWR Calculation
	return((fwd * fwd) / 50);
}
float getHeat()     //Get heat of RA60H1317M1A measured by DS18B20, should be smaller than 50
{
	ds18b20.requestTemperatures();
	return(ds18b20.getTempCByIndex(0));
}
void setTXGain(float gain)  //Set Gain of RA60H1317M1A via MCP4728 (VOUTD)
{
	if(gain == 0)
	{
		mcp4728.setChannelValue(MCP4728_CHANNEL_D, 0);
	}
	else
	{
		//ToDo Formula
		mcp4728.setChannelValue(MCP4728_CHANNEL_D, 4095);
	}
}
void setAttenuation(MCP4728_channel_t channel, float att) //Attenuation of U11 (RX) or U12 (TX) (F2255NLGK)
{
	//Note that the PGA-103 has a constant gain of 25dB for RX path! RX_ATTENUATOR_CHANNEL MCP4728_CHANNEL_A
	//Note that an 6dB Attenuator is assembled on board for TX path! TX_ATTENUATOR_CHANNEL MCP4728_CHANNEL_B
	
	if(att > 35)
	{
		mcp4728.setChannelValue(channel, 2048);
	}
	else if(att < 2.5)
	{
		mcp4728.setChannelValue(channel, 0);
	}
	else
	{
		att = att * (-1);
		mcp4728.setChannelValue(channel, (uint16_t)(4096 * (-0.03692 * att + 0.60769) / 5));  //0.7V (-2.5dB) to 1.9V (-35dB), magic numbers are calculated according to the attenuation curve of the datasheet!
	}
}
void switchPAtoTX(uint8_t TX_Att, uint8_t TX_Gain)
{
	if(monitorPA == 1)
	{
		//turn OFF RX:
		mcp4728.setChannelValue(MCP4728_CHANNEL_C, 0);    //turn RX Path in TRX Switch OFF (VOUTC to LOW)
		PORTB &= ~(1 << PGA_SHDN);                        //turn OFF PGA RX-Amp. and RX Switch (D13 to LOW)
		setAttenuation(RX_ATTENUATOR_CHANNEL, 36);        //set full attenuation of U11 (F2255NLGK)
		
		delay(100);     //100ms settling time
		
		//turn ON TX:
		PORTD &= ~(1 << PTT_OUT);                         //turn TX Path in TRX Switch ON (D7 LOW)
		PORTB |= (1 << TX_SW);                            //turn on TX Switch (D12 to HIGH)
		setAttenuation(TX_ATTENUATOR_CHANNEL, TX_Att);    //set attenuation of U12 (F2255NLGK)
		setTXGain(TX_Gain);                               //Set TX gain (VGG (Gate) voltage) of RA60H1317M1A
		delay(100);                                       //100ms settling time
	}
	else
	{
		switchPAtoRX(rxAtt);
	}
}
void switchPAtoRX(float RX_Att)
{
	//turn OFF TX:
	setTXGain(0);                                       //Set gain of RA60H1317M1A to 0
	setAttenuation(TX_ATTENUATOR_CHANNEL, 36);          //set full attenuation of U12 (F2255NLGK)
	PORTB &= ~(1 << TX_SW);                             //turn OFF TX Switch (D12 to LOW)
	PORTD |= (1 << PTT_OUT);                            //turn TX Path in TRX Switch OFF (D7 HIGH)
	delay(100);                                         //100ms settling time
	
	//turn ON RX
	mcp4728.setChannelValue(MCP4728_CHANNEL_C, 4095);   //turn RX Path in TRX Switch ON (VOUTC to HIGH)
	PORTB |= (1 << PGA_SHDN);                           //turn ON PGA RX-Amp. and RX Switch (D13 to HIGH)
	setAttenuation(RX_ATTENUATOR_CHANNEL, RX_Att);      //set attenuation of U11 (F2255NLGK)
	delay(100);                                         //100ms settling time
}
void verifyPA()   //Shutdown RA60H1317M1A according to SWR, VDD and Heat
{
	float vdd = getVDD();
	float swr = getSWR();
	float heat = getHeat();
	Serial.print("VDD: ");
	Serial.println(vdd);
	Serial.print("SWR: ");
	Serial.println(swr);
	Serial.print("HEAT: ");
	Serial.println(heat);
	
	//if((vdd >= VDD_LOWER_LIMIT && vdd <= VDD_UPPER_LIMIT) && (swr >= SWR_LOWER_LIMIT && swr <= SWR_UPPER_LIMIT) && heat <= HEAT_UPPER_LIMIT)
	if((vdd >= VDD_LOWER_LIMIT && vdd <= VDD_UPPER_LIMIT) && heat <= HEAT_UPPER_LIMIT)
	{
		monitorPA = 1;  //PA is OK!
	}
	else
	{
		monitorPA = 0;
		if(TX == 1)
		{
			switchPAtoRX(rxAtt);
		}
	}
}

//Display Methods
void displayInit()
{
	delay(100);                                 // This delay is needed to let the display to initialize
	if(!display.begin(SSD1306_SWITCHCAPVCC, 0x3C)){ // Address 0x3C for 128x32
		Serial.println(F("SSD1306 allocation failed"));
		while (1){
			delay(10);
		}
	}
	display.clearDisplay();                     // Clear the buffer
	display.setTextColor(WHITE);                // Set color of the text
	//display.setRotation(2);                     // Set orientation. Goes from 0, 1, 2 or 3
	display.setTextWrap(false);                 // By default, long lines of text are set to automatically "wrap" back to the leftmost column. 
	// To override this behavior (so text will run off the right side of the display - useful for scrolling marquee effects), use setTextWrap(false).
	display.dim(0);                             //Set brightness (0 is maximun and 1 is a little dim)
	display.setFont(&FreeMono9pt7b);            // Set a custom font
	display.setTextSize(0);                     // Set text size. We are using a custom font so you should always use the text size of 0
	display.display();                          // Print everything we set previously
}
void displayStartScreen()
{
	display.setCursor(10, 15);                  // (x,y)
	display.println("60W FM TRX");              // Text or value to print
	display.setCursor(10, 35);                  // (x,y)
	display.println("by OE3SDE");               // Text or value to print
	display.setCursor(10, 55);                  // (x,y)
	display.println("S. Dorrer");               // Text or value to print
	display.display();                          // Print everything we set previously
}
void refreshDisplay()
{
	display.clearDisplay();
	
	// TX / RX
	if(TX == 1)
	{
		display.setCursor(10, 15);                  // (x,y)
		display.println("TX");
	}
	else if(TX == 0)
	{
		display.setCursor(10, 15);                  // (x,y)
		display.println("RX");
	}
	
	// Power
	display.setCursor(50, 15);                    // (x,y)
	display.println("PWR:.5W");
	
	// Frequency
	display.setCursor(30, 35);                    // (x,y)
	display.println(dra818.txFreq);
	
	//Values
	display.setCursor(2, 55);                  // (x,y)
	display.print(getVDD());                    // Text or value to print
	display.print(" ");
	display.print(getHeat());                  // Text or value to print  
	display.display();
}
//----------------------------------------------------------------------------

//Setup (Initialize hardware)
void setup()
{
	//HW Init
	IO_Init();
	INT0_Init();
	INT1_Init();
	Timer1_COMPA_Init();
	Timer2_COMPA_Init();
	ADC_Init();
	
	// OLED Init
	displayInit();
	displayStartScreen();
	
	//DRA818V Init
	DRA818V_Init();
	
	//DS18B20 Init
	ds18b20.begin(); 
	
	//MCP4728 Init
	if(!mcp4728.begin()){
		Serial.println("Failed to find MCP4728 chip!");
		while (1){
			delay(10);
		}
	}
	
	//RA60H1317M1A
	verifyPA();
	switchPAtoRX(rxAtt); //standard attenuation of 10dB
	
	sei(); //enable global interrupts
	
	//Reads the initial state of the outputA (E2)
	aLastState = digitalRead(E2); 	// #PJN: wondering why you're now using the digitalRead() because output handling is done via direct register access - it's however OK anyways
	
	display.clearDisplay();
	display.display();
}
//----------------------------------------------------------------------------

//Loop
void loop()
{
	if(timer1Count >= 500)
	{
		timer1Count = 0;
		verifyPA();
		refreshDisplay();
	}
	
	if(TX == 1 && switched == 1)
	{
		switchPAtoTX(txAtt, txGain);    //PA PCB is transmitting now!
		//PORTD &= ~(1 << PTT_OUT);     //DRA818V TX (done in switchPAtoTX)
		switched = 0;
	}
	else if (TX == 0 && switched == 1)
	{
		//PORTD |= (1 << PTT_OUT);      //DRA818V RX (done in switchPAtoRX)
		switchPAtoRX(rxAtt);            //PA PCB is receiving now!
		switched = 0;
	}
	
	// #PJN: Because of the above code you'll get a lot of jitter when reading the encoder below
	aState = digitalRead(E2); // Reads the "current" state of the outputA
	// If the previous and the current state of the outputA are different, that means a Pulse has occured
	if (aState != aLastState)
	{     
		// If the outputB state is different to the outputA state, that means the encoder is rotating clockwise
		if (digitalRead(E1) != aState) 
		{ 
			counter++;
		} 
		else 
		{
			counter--;
		}
	} 
	aLastState = aState; // Updates the previous state of the outputA with the current state
}
//----------------------------------------------------------------------------

//ISP (Interrupt Service Routine) 
ISR(TIMER1_COMPA_vect)  // every 1ms
{
	timer1Count++;
}

ISR(TIMER2_COMPA_vect) // execute by TIMER2 CTC
{
	static uint16_t timer2Count = 0; //only at first ISR call
	
	timer2Count++; //inkrementieren
	
	if(timer2Count == 4000) //256ms sperren
	{
		//enable INT0
		EIMSK |= (1 << INT0); //Ext. Int0 on
		
		//enable INT1
		EIMSK |= (1 << INT1); //Ext. Int0 on
		
		//clear INTF0
		EIFR |= (1 << INTF0);
		
		//clear INTF1
		EIFR |= (1 << INTF1);
		
		//disable TIMER2 compare interrupt
		TIMSK2 &= ~(1 << OCIE2A);
		
		//reset of counter value
		timer2Count = 0;
	}
}

ISR(INT0_vect)  //Rotary Encoder Switch
{
	static uint8_t encoderPressedCounter = 1; //only at first ISR call
	
	uint8_t temp = SREG; //Store SREG
	
	encoderPressedCounter++;
	
	//DEBOUNCE
	//disable INT0
	EIMSK &= ~(1 << INT0); //Ext. INT0 off
	//enable TIMER2 compare interrupt
	TIMSK2 |= (1 << OCIE2A);
	
	switch (encoderPressedCounter) 
	{
		case 1:
		// #PJN: Off topic: channel spacing on 2m is 12.5 kHz ... maybe just increase/decrease frequency values instead of fixed ones?
		dra818.txFreq = 145.5000;   //TX-Frequency in MHz
		dra818.rxFreq = 145.5000;   //RX-Frequency in MHz
		DRA818V_setGroup();
		break;
		case 2:
		dra818.txFreq = 144.2000;   //TX-Frequency in MHz
		dra818.rxFreq = 144.2000;   //RX-Frequency in MHz
		DRA818V_setGroup();
		break;
		case 3:
		dra818.txFreq = 144.5000;   //TX-Frequency in MHz
		dra818.rxFreq = 144.5000;   //RX-Frequency in MHz
		DRA818V_setGroup();
		break;
		case 4:
		dra818.txFreq = 145.2000;   //TX-Frequency in MHz
		dra818.rxFreq = 145.2000;   //RX-Frequency in MHz
		DRA818V_setGroup();
		break;
		case 5:
		dra818.txFreq = 145.0000;   //TX-Frequency in MHz
		dra818.rxFreq = 145.0000;   //RX-Frequency in MHz
		DRA818V_setGroup();
		encoderPressedCounter = 0;
		break;
		default:
		encoderPressedCounter = 0;
		break;
	}
	
	SREG = temp;
}

ISR(INT1_vect) //PTT-IN (active LOW)
{
	//DEBOUNCE
	//disable INT1
	//EIMSK &= ~(1 << INT1); //Ext. INT1 off
	//enable TIMER2 compare interrupt
	//TIMSK2 |= (1 << OCIE2A);
	
	switched = 1;
	
	if(!(PIND & (1 << PTT_IN)))         //PTT_IN changed to LOW
	{
		//switchPAtoTX(txAtt, txGain);    //PA PCB is transmitting now!
		//PORTD &= ~(1 << PTT_OUT);       //DRA818V TX (done in switchPAtoTX)
		TX = 1;
	}
	else if (PIND & (1 << PTT_IN))      //PTT_IN changed to HIGH
	{
		//PORTD |= (1 << PTT_OUT);        //DRA818V RX (done in switchPAtoRX)
		//switchPAtoRX(rxAtt);            //PA PCB is receiving now!
		TX = 0;
	}
}
//----------------------------------------------------------------------------
//End of Code!
\end{lstlisting}